\documentclass[conference, onecolumn]{IEEEtran}
\IEEEoverridecommandlockouts
% The preceding line is only needed to identify funding in the first footnote. If that is unneeded, please comment it out.
\usepackage{cite}
\usepackage{amsmath,amssymb,amsfonts}
\usepackage{algorithmic}
\usepackage{graphicx}
\usepackage{textcomp}
\usepackage{xcolor}
\def\BibTeX{{\rm B\kern-.05em{\sc i\kern-.025em b}\kern-.08em
    T\kern-.1667em\lower.7ex\hbox{E}\kern-.125emX}}
\begin{document}

\title{Architectural Study of Apache Web Server\\
% {\footnotesize \textsuperscript{*}Note: Sub-titles are not captured in Xplore and
% should not be used}
% \thanks{Identify applicable funding agency here. If none, delete this.}
}

\author{\IEEEauthorblockN{VJS Pranavasri}
\IEEEauthorblockA{\textit{Interational Institute of Information Technology, Hyderabad} \\
Hyderabad, India \\
vjs.pranavasri@research.iiit.ac.in}
}

\maketitle

\begin{abstract}
This document is a paper that studies the architecture of Apache webserver in detail, and tries to find out any significant ideas or flaws.
\end{abstract}

\begin{IEEEkeywords}
    webserver, apache, web
\end{IEEEkeywords}

\section{Introduction}
\color{blue}
Give a brief (1-2 page) introduction to the system. Indicate what problem the system purports to solve, sketch its history, and name the key players (organizations and individuals) involved in its definition, design, development and evolution. Include any other basic information that provides useful context for the reader in understanding the remaining sections.
\color{black}

\section{Relationship to the Architecture Business Cycle}
\color{blue}
This section will clearly identify the context or market for the system, the stakeholders involved, the technical environment at the time the system was conceived and the experience of the architects or designers. What were the motives of those who proposed or commissioned the system? What were the driving forces that informed the architects' decisions? Include a diagram of the ABC such as the one explained in class. There is no suggested size for this section as it is heavily dependent on the particular system; just make sure it doesn't crowd out the sections that follow.
\color{black}

\section{Requirements and Qualities}
\color{blue}
What were the key functional requirements? What were the key quality attributes that the system had to exhibit? Why? What was the relative ranking of these attributes? Why? If feasible, give examples of how the quality attributes are visible in the delivered.
\color{black}

\section{Architectural Solution}
\color{blue}
This section is the most important part of the report, and should clearly describe the architecture and how it achieved (or failed to achieve) its functional and quality goals.
What are the key aspects of resulting architecture? What views are most appropriate for illustrating the architects' approach to achieving the functional and quality requirements? What is the content of these views (graphics should appropriately be provided here). Be sure to include a discussion of any alternatives considered and/or the rationale for the architectural choices made. Identify specific patterns and/or tactics found in the architecture.
\color{black}

\section{Summary}
\color{blue}
Briefly reflect on the system you researched. Did it meet its goals? Is it exemplary in some respects? What are its strengths and weaknesses relative to the desired functionality and qualities? Is there anything of note about the system that is not described in one of the previous sections?
Sample citation\cite{IEEEexample:confwithadddays}
\color{black}

% \section{References} taken care of by the bibliography
\bibliographystyle{IEEEtran}
\bibliography{references}

\appendix
Just in case I want to add an appendix

\section*{Effort}
VJS Pranavasri : X hours

\vspace{12pt}
\color{red}
This is just the format for the report. The actual report is yet to be added.

\end{document}
